\documentclass{article}
\usepackage{amsmath,amssymb,pxfonts,mathpazo,ulem}
\usepackage[utf8]{inputenc}
\usepackage[colorlinks]{hyperref}

\newcommand{\lnk}[2]{\href{#1}{\textcolor[rgb]{1.0,0.0,0.0}{#2}}}

\title{Data Science in Ten Minutes\footnote{for extremely large values of ten}}
\author{Spencer Tipping}

\begin{document}
  \maketitle
  \tableofcontents

  \section{Introduction}
  {\bf Data science is not machine learning.}

  \newpage
  \section{Linux}
  Oh yes, we are totally going here. Here's why.

  Backend programs and processing pipelines and stuff (basically, ``big data''
  things) operate entirely by talking to the kernel, which, in big-data world,
  is usually Linux; and this is true regardless of the language, libraries, and
  framework(s) you're using. You can always throw more hardware at a
  problem\footnote{Until you can't}, but if you understand system-level
  programming you'll often have a better/cheaper option.

  Some quick background reading if you need it for the homework/curiosity:

  \begin{itemize}
    \item \lnk{https://github.com/spencertipping/shell-tutorial}
              {How to write a UNIX shell}
    \item \lnk{https://github.com/spencertipping/jit-tutorial}
              {How to write a JIT compiler}
    \item \lnk{http://manpages.ubuntu.com/manpages/xenial/man5/elf.5.html}
              {ELF executable binary spec}
    \item \lnk{https://www.intel.com/content/dam/www/public/us/en/documents/manuals/64-ia-32-architectures-software-developer-instruction-set-reference-manual-325383.pdf}
              {Intel machine code documentation}
  \end{itemize}

  \subsection{Virtual memory}
  This is one of the two things that comes up a lot in data science
  infrastructure. Basically, any memory address you can see is virtualized and
  may not be resident in the RAM chips in your machine. The kernel talks to the
  \lnk{https://en.wikipedia.org/wiki/Memory_management_unit}{MMU hardware} to
  maintain the mapping between software and hardware pages, and when the virtual
  page set overflows physical memory the kernel swaps them to disk, usually with
  an \lnk{https://en.wikipedia.org/wiki/Cache_replacement_policies}{LRU}
  strategy.

  Here's where things get interesting. Linux (and any other server OS) gives you
  a {\tt mmap} system call to request that the kernel map pages into your
  program's address space. {\tt mmap}, however, has some interesting options:

  \begin{itemize}
    \item \verb|MAP_SHARED|:~map the region into multiple programs' address
          spaces (this reuses the same physical page across processes)
    \item \verb|MAP_FILE|:~map the region using data from a file; then the
          kernel will load file data when a page fault occurs
  \end{itemize}

  \verb|MAP_FILE| is sort of like saying ``swap this region to a specific file,
  rather than the shared swapfile you'd normally use.'' The implication is
  important, though:~{\it all memory mappings go through the same page
  allocation cache,} and any page fault has the potential to block your program
  on disk IO. Clever data structures like
  \lnk{https://en.wikipedia.org/wiki/Bloom_filter}{Bloom filters},
  \lnk{https://en.wikipedia.org/wiki/Count\%E2\%80\%93min_sketch}{Count-min
  sketches}, and so forth are all designed to give you a way to trade various
  degrees of accuracy for a much smaller memory footprint.

  Sometimes you won't have any good options within the confines of physical RAM,
  so you'll end up using IO devices to supply data; then the challenge becomes
  optimizing for those IO devices (SSDs are different from HDDs, for instance).
  I'll get to some specifics later on when we talk about sorting, joins, and
  compression.

  \subsection{File descriptors}
  This is the other thing you need to know about.

  Programs don't typically use {\tt mmap} for general-purpose IO. It's more
  idiomatic, and sometimes faster, to use {\tt read} and {\tt write} on a file
  descriptor. Internally, these functions ask the kernel to copy memory from an
  underyling file/socket/pipe/etc into mapped pages in the address space. The
  advantage is cache locality:~you can {\tt read} a small amount of stuff into a
  buffer, process the buffer, and then reuse that buffer for the next {\tt
  read}. Cache locality does matter; for example:

  \begin{verbatim}
# small block size: great cache locality, too much system calling overhead
$ dd if=/dev/zero count=262144 bs=32768 of=/dev/null
262144+0 records in
262144+0 records out
8589934592 bytes (8.6 GB, 8.0 GiB) copied, 0.774034 s, 11.1 GB/s

# medium block size: great cache locality, insignificant syscall overhead
$ dd if=/dev/zero count=8192 bs=1048576 of=/dev/null
8192+0 records in
8192+0 records out
8589934592 bytes (8.6 GB, 8.0 GiB) copied, 0.574597 s, 14.9 GB/s

# large block size: cache overflow, insignificant syscall overhead
$ dd if=/dev/zero count=2048 bs=$((1048576 * 4)) of=/dev/null
2048+0 records in
2048+0 records out
8589934592 bytes (8.6 GB, 8.0 GiB) copied, 1.14022 s, 7.5 GB/s
  \end{verbatim}

  \subsection{Pulling this together:~let's write a program}
  ...in machine language.

  \subsection{Homework (if that's your thing)}
  \begin{enumerate}
    \item Write an ELF Linux executable that prints {\tt hello world} to stdout,
          then exits successfully.
  \end{enumerate}
\end{document}
