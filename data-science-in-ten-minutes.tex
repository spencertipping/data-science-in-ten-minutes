\documentclass{article}
\usepackage{amsmath,amssymb,pxfonts,mathpazo,ulem}
\usepackage[utf8]{inputenc}
\usepackage[colorlinks]{hyperref}

\newcommand{\lnk}[2]{\href{#1}{\textcolor[rgb]{1.0,0.0,0.0}{#2}}}

\title{Data Science in Ten Minutes\footnote{for extremely large values of ten}}
\author{Spencer Tipping}

\begin{document}
  \maketitle
  \tableofcontents

  \section{Introduction}
  {\bf Data science is not machine learning.} There is no machine learning in
  this guide, because machine learning is the wrong answer to most problems
  you're likely to run into. I've never deployed a machine learning system,
  although I've debugged several misbehaving ones (and will cover how to do
  that).

  A lot of the code examples assume you're running Linux because Linux is the
  go-to platform for medium/big data processing. It's also ideal for small data
  due to optimizations in core utilities like {\tt sort}.\footnote{GNU {\tt sort}
  will compress temporary files, unlike the {\tt sort} that ships with Mac OSX
  for instance.} If you're running a non-Linux OS and want to try stuff, you
  have a few options:

  \begin{itemize}
    \item Install \lnk{https://docker.com}{Docker} and create a container from
          the {\tt ubuntu:18.04} image or similar (I'll assume you have this
          setup)
    \item Run a VM like \lnk{https://www.virtualbox.org/}{VirtualBox} and
          install \lnk{https://www.ubuntu.com/desktop}{Ubuntu Desktop} inside it
    \item Rent a \lnk{https://aws.amazon.com/free/}{free tier} Amazon EC2
          instance (you can use the free-tier {\tt t2.nano} or {\tt t2.micro}
          for data science, and they're ideal for learning because they're
          resource-constrained)
    \item Buy a
          \lnk{https://github.com/spencertipping/www/blob/master/datacenter.md}{cheap
          rack server from eBay} and drop Ubuntu Server on it
  \end{itemize}

  Realistically, you probably won't want to use OSX or Windows for distributed
  programming:~your binaries won't be portable to cluster machines, and you'll
  be fighting with things like case-insensitive filesystems, slower core
  utilities, and general inconsistencies that will increase debugging
  time.\footnote{All of this is a non-issue for JVM processes, e.g.~Hadoop and
  Spark, but a lot of data science is best done with native tools that use less
  memory and have lower iteration time.}

  \subsection{Examples}
  This guide comes with
  \lnk{https://github.com/spencertipping/data-science-in-ten-minutes/tree/master/example}{example
  code} in the original repository. Where necessary, I've included package
  installation commands required to install dependencies on Ubuntu 18.04.

  \newpage
  \section{Linux}
  Oh yes, we are totally going here. Here's why.

  Backend programs and processing pipelines and stuff (basically, ``big data''
  things) operate entirely by talking to the kernel, which, in big-data world,
  is usually Linux; and this is true regardless of the language, libraries, and
  framework(s) you're using. You can always throw more hardware at a
  problem\footnote{Until you can't}, but if you understand system-level
  programming you'll often have a better/cheaper option.

  Some quick background reading if you need it for the homework/curiosity:

  \begin{itemize}
    \item \lnk{https://github.com/spencertipping/shell-tutorial}
              {How to write a UNIX shell}
    \item \lnk{https://github.com/spencertipping/jit-tutorial}
              {How to write a JIT compiler}
    \item \lnk{http://manpages.ubuntu.com/manpages/xenial/man5/elf.5.html}
              {ELF executable binary spec} (more readable version
              \lnk{https://en.wikipedia.org/wiki/Executable_and_Linkable_Format}
              {on Wikipedia}, and
              \lnk{https://stackoverflow.com/questions/2427011/what-is-the-difference-between-elf-files-and-bin-files}
                  {this StackOverflow answer} may be useful)
    \item \lnk{https://www.intel.com/content/dam/www/public/us/en/documents/manuals/64-ia-32-architectures-software-developer-instruction-set-reference-manual-325383.pdf}
              {Intel machine code documentation}
  \end{itemize}

  \subsection{Virtual memory}
  This is one of the two things that comes up a lot in data science
  infrastructure. Basically, any memory address you can see is virtualized and
  may not be resident in the RAM chips in your machine. The kernel talks to the
  \lnk{https://en.wikipedia.org/wiki/Memory_management_unit}{MMU hardware} to
  maintain the mapping between software and hardware pages, and when the virtual
  page set overflows physical memory the kernel swaps them to disk, usually with
  an \lnk{https://en.wikipedia.org/wiki/Cache_replacement_policies}{LRU}
  strategy.

  Here's where things get interesting. Linux (and any other server OS) gives you
  a {\tt mmap} system call to request that the kernel map pages into your
  program's address space. {\tt mmap}, however, has some interesting options:

  \begin{itemize}
    \item \verb|MAP_SHARED|:~map the region into multiple programs' address
          spaces (this reuses the same physical page across processes)
    \item \verb|MAP_FILE|:~map the region using data from a file; then the
          kernel will load file data when a page fault occurs
  \end{itemize}

  \verb|MAP_FILE| is sort of like saying ``swap this region to a specific file,
  rather than the shared swapfile you'd normally use.'' The implication is
  important, though:~{\it all memory mappings go through the same page
  allocation cache,} and any page fault has the potential to block your program
  on disk IO. Clever data structures like
  \lnk{https://en.wikipedia.org/wiki/Bloom_filter}{Bloom filters},
  \lnk{https://en.wikipedia.org/wiki/Count\%E2\%80\%93min_sketch}{Count-min
  sketches}, and so forth are all designed to give you a way to trade various
  degrees of accuracy for a much smaller memory footprint.

  Sometimes you won't have any good options within the confines of physical RAM,
  so you'll end up using IO devices to supply data; then the challenge becomes
  optimizing for those IO devices (SSDs are different from HDDs, for instance).
  I'll get to some specifics later on when we talk about sorting, joins, and
  compression.

  \subsection{File descriptors}
  This is the other thing you need to know about.

  Programs don't typically use {\tt mmap} for general-purpose IO. It's more
  idiomatic, and sometimes faster, to use {\tt read} and {\tt write} on a file
  descriptor. Internally, these functions ask the kernel to copy memory from an
  underyling file/socket/pipe/etc into mapped pages in the address space. The
  advantage is cache locality:~you can {\tt read} a small amount of stuff into a
  buffer, process the buffer, and then reuse that buffer for the next {\tt
  read}. Cache locality does matter; for example:

  \begin{verbatim}
# small block size: great cache locality, too much system calling overhead
$ dd if=/dev/zero count=262144 bs=32768 of=/dev/null
262144+0 records in
262144+0 records out
8589934592 bytes (8.6 GB, 8.0 GiB) copied, 0.774034 s, 11.1 GB/s

# medium block size: great cache locality, insignificant syscall overhead
$ dd if=/dev/zero count=8192 bs=1048576 of=/dev/null
8192+0 records in
8192+0 records out
8589934592 bytes (8.6 GB, 8.0 GiB) copied, 0.574597 s, 14.9 GB/s

# large block size: cache overflow, insignificant syscall overhead
$ dd if=/dev/zero count=2048 bs=$((1048576 * 4)) of=/dev/null
2048+0 records in
2048+0 records out
8589934592 bytes (8.6 GB, 8.0 GiB) copied, 1.14022 s, 7.5 GB/s\end{verbatim}

  \subsection{Pulling this together:~let's write a program}
  ...in machine language. For simplicity, let's write one that prints {\tt hello
  world} and then exits successfully (with code 0).

  This is also a good opportunity to talk about how we might generate and work
  with binary data with things like fixed offsets. Two simple functions for this
  are \lnk{https://perldoc.perl.org/perlpacktut.html}{\tt pack()} and {\tt
  unpack()}, variants of which ship with both Perl and Ruby.

  The first part of any Linux executable is the ELF header, usually followed
  directly by a program header; here's what those look like as C structs for
  64-bit executables (reformatted slightly, and with docs for readability):

  \begin{verbatim}
typedef struct {
  unsigned char e_ident[16];    // 0x7f, 'E', 'L', 'F', ...
  uint16_t      e_type;         // ET_EXEC = 2 for executable files
  uint16_t      e_machine;      // EM_X86_64 = 62 for AMD64 architecture
  uint32_t      e_version;      // EV_CURRENT = 1
  uint64_t      e_entry;        // virtual address of first instruction
  uint64_t      e_phoff;        // file offset of first program header
  uint64_t      e_shoff;        // file offset of first section header
  uint32_t      e_flags;        // always zero
  uint16_t      e_ehsize;       // size of the ELF header struct (this one)
  uint16_t      e_phentsize;    // size of a program header struct
  uint16_t      e_phnum;        // number of program header structs
  uint16_t      e_shentsize;    // size of a section header struct
  uint16_t      e_shnum;        // number of section header structs
  uint16_t      e_shstrndx;     // string table linkage
} Elf64_Ehdr;

typedef struct {
  uint32_t p_type;              // the purpose of the mapping
  uint32_t p_flags;             // permissions for the mapped pages (rwx)
  uint64_t p_offset;            // file offset of the first byte of data
  uint64_t p_vaddr;             // virtual memory offset of the data (NB below)
  uint64_t p_paddr;             // physical memory offset (usually zero)
  uint64_t p_filesz;            // number of bytes from the file
  uint64_t p_memsz;             // number of bytes to be mapped into memory
  uint64_t p_align;             // segment alignment
} Elf64_Phdr;\end{verbatim}

  If we want a very minimal executable, here's how we might write these headers
  from Perl:

  \begin{verbatim}
# elf-header.pl: emit an ELF binary and program header to stdout
use strict;
use warnings;

print pack('C16 SSL',
           0x7f, ord 'E', ord 'L', ord 'F',
           2, 1, 1, 0,
           0, 0, 0, 0,
           0, 0, 0, 0,

           2,                           # e_type    = ET_EXEC
           62,                          # e_machine = EM_X86_64
           1)                           # e_version = EV_CURRENT

    . pack('QQQ',
           0x400078,                    # e_entry = 0x400078
           64,                          # e_phoff
           0)                           # e_shoff

    . pack('LSS SSSS',
           0,                           # e_flags
           64,                          # e_ehsize
           56,                          # e_phentsize
           1,                           # e_phnum

           0,                           # e_shentsize
           0,                           # e_shnum
           0)                           # e_shstrndx

    . pack('LLQQQQQQ',
           1,                           # p_type = PT_LOAD (map a region)
           7,                           # p_flags = R|W|X
           0,                           # p_offset (must be page-aligned)
           0x400000,                    # p_vaddr
           0,                           # p_paddr
           0x1000,                      # p_filesz: 4KB
           0x1000,                      # p_memsz: 4KB
           0x1000);                     # p_align: 4KB\end{verbatim}

  Now we can generate the ELF header:

  \begin{verbatim}
$ sudo apt install perl                 # if perl is missing
$ perl elf-header.pl > elf-header\end{verbatim}

  You can verify that the header is correct using {\tt file}, which reads magic
  numbers and tells you about the format of things:

  \begin{verbatim}
$ sudo apt install file
$ file elf-header
elf-header: ELF 64-bit LSB executable, x86-64, version 1 (SYSV), statically
linked, corrupted section header size\end{verbatim}

  Awesome, now let's get into the machine code.

  \subsection{{\tt hello world} in machine code}
  The first thing to note is that we're talking directly to the kernel here. Our
  ELF header above is very minimalistic, with no linker instructions or anything
  else to complicate things. So we have none of the usual libc functions like
  {\tt printf} or {\tt exit}; it's up to us to define those in terms of Linux
  system calls.

  There are two calls we'll use for this:~{\tt write(2)} and {\tt exit(2)}. You
  can view the documentation for these using {\tt man}:

  \begin{verbatim}
$ sudo apt install man manpages-dev
$ man 2 write
$ man 2 exit\end{verbatim}

  We'll need to pass arguments in registers to match the
  \lnk{https://stackoverflow.com/questions/2535989/what-are-the-calling-conventions-for-unix-linux-system-calls-on-i386-and-x86-6}
      {kernel calling convention}. I'll spare you the gory details and cut
  straight to the machine code:

  \begin{verbatim}
# elf-hello.pl: emit machine code for hello world, then exit successfully
use strict;
use warnings;
print pack('H*', join '',
  '4831c0',                             # xorq %rax, %rax
  'b001',                               # movb $01, %al (1 = write syscall)
  'e80c000000',                         # call %rip+12 (jump over the message)
  unpack('H*', "hello world\n"),        # the message
  '5e',                                 # pop message into %rsi (buf arg)
  'ba0c000000',                         # movl $12, %rdx (len arg)
  '48c7c701000000',                     # movl $1, %rdi (fd arg)
  '0f05',                               # syscall instruction

  '4831c0',                             # xorq %rax, %rax
  'b03c',                               # movb $3c, %rax (3c = exit syscall)
  '4831ff',                             # xorq %rdi, %rdi (exit code arg)
  '0f05');                              # syscall instruction\end{verbatim}

  Now we can build the full executable, verify it, and run:

  \begin{verbatim}
$ sudo apt install nasm
$ perl elf-hello.pl | cat elf-header - > elf-hello
$ chmod 755 elf-hello
$ tail -c+121 elf-hello | ndisasm -b 64 -
00000000  4831C0            xor rax,rax
00000003  B001              mov al,0x1
00000005  E80C000000        call 0x16
0000000A  68656C6C6F        push qword 0x6f6c6c65   # corruption from message
0000000F  20776F            and [rdi+0x6f],dh       # (which we jumped over)
00000012  726C              jc 0x80
00000014  640A5EBA          or bl,[fs:rsi-0x46]
00000018  0C00              or al,0x0
0000001A  0000              add [rax],al
0000001C  48C7C701000000    mov rdi,0x1             # now we're back on track
00000023  0F05              syscall
00000025  4831C0            xor rax,rax
00000028  B03C              mov al,0x3c
0000002A  4831FF            xor rdi,rdi
0000002D  0F05              syscall
  \end{verbatim}

  The moment we've been waiting for:

  \begin{verbatim}
$ ./elf-hello
hello world
$ echo $?                               # check exit status
0
  \end{verbatim}

  Whether through libraries, JIT, or anything else, this is the exact mechanism
  being used by any program that performs IO of any sort:~the program lives in a
  completely virtual world and interacts with the kernel using the {\tt 0f05}
  syscall instruction, referring to virtual addresses in the process.

  \subsection{Homework (if that's your thing)}
  \begin{enumerate}
    \item Write an ELF Linux executable that consumes data from {\tt stdin} and
          writes that data to {\tt stdout}, then exits successfully. In other
          words, {\tt cat} without file support.
  \end{enumerate}
\end{document}
