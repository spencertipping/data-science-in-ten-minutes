\section{Wikipedia}
You can
\lnk{https://en.wikipedia.org/wiki/Wikipedia:Database_download}{download the
full English language Wikipedia} as a giant bzip2-compressed XML file. For
reference,
\lnk{http://itorrents.org/torrent/D567CE8E2EC4792A99197FB61DEAEBD70ADD97C0.torrent}{here's
the exact torrent file I downloaded}. It's about 14GB compressed, which is not
big data by any means; you could process this on a Raspberry Pi with a 64GB SD
card if you wanted to.\footnote{Be careful with SD cards and Flash storage in
general; if you write the memory too many times you'll destroy the drive. I'll
mention this hazard anytime I have an IO-intensive process.}

While that's downloading, let's take a moment to talk about compression formats.

\subsection{Compression}
There are a few standard, general-purpose data compressors you're likely to
encounter regularly:\footnote{I ran these tests by compressing an infinite
stream of copies of \lnk{https://github.com/spencertipping/ni}{the ni
repository}, which is large enough to overflow any buffers used by these
algorithms. The exact script template was {\tt ni ::self[//ni] npself zx9 zn}.}

TODO:~use wikipedia data for the table below, obviously

\begin{table}[ht]
\begin{tabular}{llll}
  Compressor  & Compression speed & Decompression speed & Efficiency \\
  \hline
  {\tt xz}    & 4MB/s             & 200MB/s TODO        & High \\
  {\tt bzip2} & 8MB/s             & {\bf 24MB/s}        & High-ish \\
  {\tt gzip}  & 23MB/s            & 120MB/s             & Medium \\
  {\tt lzo}   & 200MB/s           & 300MB/s TODO        & Low \\
  {\tt lz4}   & 240MB/s           & 800MB/s TODO        & Low
\end{tabular}
\end{table}

If you take one thing away from this table, it's {\it don't use bzip2}. {\tt
bzip2} is horrible, even though the
\lnk{https://en.wikipedia.org/wiki/Burrows\%E2\%80\%93Wheeler_transform}{algorithm}
is pretty cool. If you are ever cursed with a bzip2 file, you can accelerate
decompression by parallelizing it across multiple cores using {\tt pbzip2},
which is installable under Ubuntu using {\tt sudo apt install pbzip2}.

Roughly speaking, here's how these compressors operate:\footnote{And
\lnk{http://spencertipping.com/information-theory-in-ten-minutes/information-theory-in-ten-minutes.pdf}{here's
some background on the theory}, if that's of interest}

\begin{table}[ht]
\begin{tabular}{ll}
  Compressor  & Structure \\
  \hline
  {\tt xz}    & Large-dictionary \lnk{https://en.wikipedia.org/wiki/LZ77_and_LZ78}{LZ77}
                + \lnk{https://en.wikipedia.org/wiki/Lempel\%E2\%80\%93Ziv\%E2\%80\%93Markov_chain_algorithm}{LZMA}
                + statistical prediction \\
  {\tt bzip2} & \lnk{https://en.wikipedia.org/wiki/Run-length_encoding}{RLE}
                + \lnk{https://en.wikipedia.org/wiki/Burrows\%E2\%80\%93Wheeler_transform}{BWT}
                + \lnk{https://en.wikipedia.org/wiki/Move-to-front_transform}{MTF}
                + RLE
                + \lnk{https://en.wikipedia.org/wiki/Huffman_coding}{Huffman}
                + bit-sparse \\
  {\tt gzip}  & \lnk{https://en.wikipedia.org/wiki/LZ77_and_LZ78}{LZ77}
                + \lnk{https://en.wikipedia.org/wiki/Huffman_coding}{Huffman} \\
  {\tt lzo}   & Dictionary \\
  {\tt lz4}   & Dictionary
\end{tabular}
\end{table}

It's worth knowing the broad strokes because the nature of the data will impact
compression performance, both in space and time. For example, the
\lnk{http://files.pushshift.io/reddit/}{Reddit comments dataset} contains a
bunch of identical-schema JSON objects that look roughly like this (reformatted
for readability):

\begin{verbatim}
{ "author":"CreativeTechGuyGames",
  "author_flair_css_class":null,
  "author_flair_text":null,
  "body":"You are looking to create a reddit bot? You will want to check out
          the [reddit API](https://www.reddit.com/dev/api). Many people do
          this in Python and there are many tutorials on the internet showing
          how to do so.",
  "can_gild":true,
  "controversiality":0,
  "created_utc":1506816001,
  "distinguished":null,
  "edited":false,
  "gilded":0,
  "id":"dnqik29",
  "is_submitter":false,
  "link_id":"t3_73if9c",
  "parent_id":"t1_dnqihqz",
  "permalink":"/r/learnprogramming/comments/73if9c/how_i_would_i_go_around_making_something_like_this/dnqik29/",
  "retrieved_on":1509189607,
  "score":1,
  "stickied":false,
  "subreddit":"learnprogramming",
  "subreddit_id":"t5_2r7yd" }
\end{verbatim}

A nontrivial amount of the bulk in this file is stored in the JSON field names,
so dictionary encoding alone is likely to save us a nontrivial amount of space.
LZ4 is worthwhile here just for that, and would almost certainly be faster than
reading directly from the underlying IO device.

Sometimes you need Huffman encoding, though; for example, random ASCII floats
don't have enough repetition to behave well with dictionary compressors. On my
test case {\tt gzip} gets about 4x better compression than {\tt lz4}, and that
might justify preferring it to LZ4 over slow IO devices even if it creates a CPU
bottleneck.

I generally start with {\tt gzip} at its default level and change algorithms
later if I need to.

\subsection{OK, back to Wikipedia}
So we have 14GB of bzip2 data:

\begin{verbatim}
$ ls -lh enwiki-20170820-pages-articles.xml.bz2
-rw-rw-r-- 1 114 122 14G May 20 00:38 enwiki-20170820-pages-articles.xml.bz2\end{verbatim}

What do we do with this? I'll present the next section two ways, one with
standard UNIX tools and one with \lnk{https://github.com/spencertipping/ni}{\tt ni},
a tool I wrote for data science.

In both cases we're solving the same problem:~let's build a list of articles
sorted by the fraction of web citations, as opposed to other types like books or
journals.

\subsection{Wikipedia with standard UNIX tools}
First we need a way to preview the data without decompressing the whole thing.
The simplest strategy is to decompress into {\tt less}:

\begin{verbatim}
$ bzip2 -dc enwiki* | less
<mediawiki xmlns="http://www.mediawiki.org/xml/export-0.10/"
           xmlns:xsi="http://www.w3.org/2001/XMLSchema-instance"
           xsi:schemaLocation="http://www.mediawiki.org/xml/export-0.10/
                               http://www.mediawiki.org/xml/export-0.10.xsd"
           version="0.10" xml:lang="en">
  <siteinfo>
    <sitename>Wikipedia</sitename>
    <dbname>enwiki</dbname>
    <base>https://en.wikipedia.org/wiki/Main_Page</base>
    <generator>MediaWiki 1.30.0-wmf.14</generator>
    <case>first-letter</case>
... \end{verbatim}

Paging around a bit, the basic structure looks roughly like this:

\begin{verbatim}
<page>
  <title>AccessibleComputing</title>
  <ns>0</ns>
  <id>10</id>
  <redirect title="Computer accessibility" />
  <revision>
    <id>767284433</id>
    <parentid>631144794</parentid>
    <timestamp>2017-02-25T00:30:28Z</timestamp>
    <contributor>
      <username>Godsy</username>
      <id>23257138</id>
    </contributor>
    <comment>[[Template:This is a redirect]] has been deprecated, change to [[Template:Redirect category shell]].</comment>
    <model>wikitext</model>
    <format>text/x-wiki</format>
    <text xml:space="preserve">#REDIRECT [[Computer accessibility]]

{{Redirect category shell|
{{R from move}}
{{R from CamelCase}}
{{R unprintworthy}}
}}</text>
    <sha1>ds1crfrjsn7xv73djcs4e4aq9niwanx</sha1>
  </revision>
</page> \end{verbatim}

Inline citations are in the text and look like this (normally one line; I've
reformatted here):

\begin{verbatim}
{{cite book
  |last=Dielo Trouda |authorlink=Dielo Truda
  |title=Organizational Platform of the General Union of Anarchists (Draft)
  |origyear=1926 |url=http://www.anarkismo.net/newswire.php?story_id=1000
  |accessdate=24 October 2006 |year=2006 |publisher=FdCA |location=Italy
  |archiveurl= https://web.archive.org/web/20070311013533/http://www.anarkismo.net/newswire.php?story_id=1000
  |archivedate= 11 March 2007&lt;!--Added by DASHBot--&gt;}} \end{verbatim}

Overall we have two stages to this process. The first should convert the XML
stream to a series of rows, let's say
\lnk{https://en.wikipedia.org/wiki/Tab-separated_values}{TSV} of {\tt webcount
othercount title}, one per article. So the {\tt AccessibleComputing}
not-really-article above would look like {\tt 0 0 AccessibleComputing}.

\begin{verbatim}
# wikipedia-cite-extract.pl: count citations by type per article
use strict;
use warnings;
while (<STDIN>)
{
  # Skip rows until we hit a title, which will be stored in $1
  next unless /<title>(.*)<\/title>/;

  # Save the title and read until the end of the article text, counting any
  # citations we find.
  my $title = $1;

  for (my ($web, $other) = (0, 0);
       !eof and ($_ =~ <STDIN>) !~ /<\/text/;)
  {
    /^web$/ ? ++$web : ++$other for /\{\{cite (\w+)/g;
  }
  print join("\t", $web, $other, $title), "\n";
}\end{verbatim}

This runs one line at a time and streams its output, so we can quickly
preview/debug/iterate. Here's what that looks like:

\begin{verbatim}
$ bzcat enwiki* | perl wikipedia-cite-extract.pl | less
0       0       AccessibleComputing
57      74      Anarchism
0       0       AfghanistanHistory
0       0       AfghanistanGeography
0       0       AfghanistanPeople
0       0       AfghanistanCommunications
0       0       AfghanistanTransportations
0       0       AfghanistanMilitary
0       0       AfghanistanTransnationalIssues
0       0       AssistiveTechnology
0       0       AmoeboidTaxa
18      207     Autism
0       0       AlbaniaHistory
...\end{verbatim}

OK, we want a ratio, so let's remove citation-free articles and calculate
$\textrm{web}/\textrm{total}$ as a fraction:

\begin{verbatim}
$ bzcat enwiki* \
    | perl wikipedia-cite-extract.pl \
    | perl -ane 'print join("\t", $F[0] / ($F[0] + $F[1]), @F[2..$#F]), "\n"
                   if $F[0] + $F[1]' \
    | less

0.435114503816794       Anarchism
0.08    Autism
0.566666666666667       Albedo
0.272727272727273       A
0.826086956521739       Alabama
0.181818181818182       Achilles
0.123529411764706       Abraham Lincoln
0.147540983606557       Aristotle
...\end{verbatim}

...and finally, before we write anything, let's sort the list by the fraction
using {\tt sort}. Before I do that, though, I want to talk a little about how
{\tt sort} works.

If you're sorting a stream of things, you have to store the whole stream first.
Then, once you have everything, you shuffle stuff $\log n$ times and emit the
sorted values.

This requires $O(n)$ space, of course, which is inconvenient:~you now have at
least one temporary copy of the data you're sorting. If you were running this in
a language like Python, Perl, or Ruby this would all happen in memory, which
limits the size of data you can sort using standard APIs. UNIX {\tt sort} is
different, though.

Internally, {\tt sort} keeps only a very small amount of data in memory, by
default something like 4MB at a time. Once it hits that limit, it writes the
sorted buffer to a temporary file and sorts the next one, later merging them
back from disk. GNU {\tt sort} in particular supports some extra options that
are useful for data like this:~{\tt --compress-program} and {\tt --parallel}.
{\tt --compress-program} instructs {\tt sort} to compress its temporary files,
effectively reducing the {\it disk} space-complexity of the sort to $O(k)$,
where $k$ is the compressed size of your data. This, obviously, can make a huge
difference.

So, with that said, here's the final pipeline:

\begin{verbatim}
$ sudo apt install pv pbzip2
$ pv enwiki* \
    | pbzip2 -dc \
    | perl wikipedia-cite-extract.pl \
    | perl -ane 'print join("\t", $F[0] / ($F[0] + $F[1]), @F[2..$#F]), "\n"
                   if $F[0] + $F[1]' \
    | sort -rn \
    | gzip > wiki-sorted.gz\end{verbatim}

{\tt pv} is {\tt cat}, but with a progress meter that looks like this:

\begin{verbatim}
 164MiB 0:00:31 [5.39MiB/s] [>                                 ]  1% ETA 0:42:08 \end{verbatim}

Having progress meters is crucial for long-running data jobs, if for no other
reason than to make sure it looks reasonable.

While that's running and before I get to the {\tt ni} version, let's talk about
some tools useful for performance monitoring.

\subsection{Monitoring tools}
I have a
\lnk{https://github.com/spencertipping/docker/blob/master/Dockerfile\#L20}{standard
set of things} I install on Linux boxes that includes:

\begin{itemize}
  \item {\tt htop}: {\tt top}, but better
  \item {\tt atop}: {\tt top} for CPU, memory, disk, network, etc
  \item {\tt units}: a unit-aware calculator (e.g.~{\tt 50GB/5Mbps in hours})
\end{itemize}

\subsection{Wikipedia with {\tt ni}}
\begin{verbatim}
$ sudo apt install git pbzip2 perl perl-modules
$ git clone git://github.com/spencertipping/ni
$ sudo ln -s $PWD/ni/ni /usr/bin/\end{verbatim}

{\tt ni} will preview compressed data automatically, so you can use it like a
compression-aware {\tt less} by default. It also knows to use {\tt pbzip2} if
you have it installed.

\begin{verbatim}
$ ni enwiki*          # preview data
$ ni enwiki* r/cite/  # select rows matching the regex /cite/, preview those \end{verbatim}

There's
\lnk{https://github.com/spencertipping/ni/blob/develop/doc/ni_by_example_1.md}{extensive
documentation on how {\tt ni} works}, which may be helpful to understand what
these commands do.

{\tt wikipedia-cite-extract.pl} and the following perl command can be folded
into a four-liner, and \verb/sort -rn | gzip/ becomes {\tt oz}:

\begin{verbatim}
$ ni enwiki* p'return () unless /<title>(.*)<\/title>/;
               my $t  = $1;
               my @cs = map /\{\{cite (\w+)/, ru {/<\/text>/} or return ();
               r grep(/^web$/, @cs) / @cs, $t' oz > wiki-sorted.gz\end{verbatim}

{\tt ni} monitors the data progress for you, so you can see a preview of the
data moving out of each pipeline stage as well as speed and bottleneck pressure.

Whether you use {\tt ni}, {\tt perl}, or something else, command-line data
processing is crucial to fast iteration on datasets. Compared to Hadoop/Spark,
it's far less typing and effort, instant startup and debugging, and no data
movement (and often faster; I'll cover that in more detail in later chapters).

\subsection{Homework}
\begin{enumerate}
  \item What is the tenth most common word in Wikipedia? Assume you're running
        in a memory-constrained environment.
  \item Write a simple disk-backed {\tt sort} utility in your favorite language.
  \item Given the TSV of {\tt web other title} we built above, what is the
        relative overhead of parsing integers (vs line processing + tab
        splitting) when we filter the data? You could answer this question for
        {\tt perl} or any other scripting language.
\end{enumerate}
